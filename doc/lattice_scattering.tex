\documentclass[a4paper,12pt]{article}
\usepackage[russian]{babel}
\usepackage[utf8]{inputenc}
\usepackage{amsmath}
\usepackage{graphicx}
\usepackage[perpage]{footmisc} 
\usepackage{braket}
\newcommand{\vect}[1]{\boldsymbol{#1}}

\title{Высокоэнергетическое рассеяние без разложения по парциальным волнам.}
\author{.}

\begin{document}
	
	%\maketitle
	\tableofcontents
    \pagebreak
     
	\begin{abstract}
		В данной работе получена конечномерная аппроксимация операторов рассеяния и уравнения Липпмана-Швингера путем перехода из базиса плоских волн в базис волновых пакетов. В таком представлении функция Грина свободной частицы интегрируется аналитически, что существенно упрощает численное решение уравнений.
	\end{abstract}
	\pagebreak

	\section{Введение}
Знание физических величин, характеризующих процесс рассеяния, необходимо во многих областях современной физики: феноменологии элементарных частиц, при вычислении интегралов столкновений в теории плазмы, при расчете замедления нейтронов в ядерных реакторах. Для решения возникающих при этом уравнений необходимы большие вычислительные мощности, и проведение таких расчетов на персональных компьютерах возможно лишь в небольшом числе случаев. В связи со все возрастающей доступностью параллельных вычислений (grid-сети, кластеры, суперкомпьютеры) возникает проблема разработки методов и алгоритмов решения задач рассеяния в многопоточном режиме. При этом возможно существенно увеличить число задач, поддающихся вычислению ab initio.

При энергиях порядка десятка МэВ вклад в процесс нуклон-нуклонного рассеяния даёт лишь небольшое количество парциальных амплитуд. В соответствии с этим разложение по парциальным волнам является адекватным методом расчета рассеяния. С другой стороны при энергиях порядка сотен МэВ и более возбуждается большие угловые моменты, которые дают вклад в амплитуду рассеяния. Такие вклады являются сильно осциллирующими функциями угла рассеяния при том, что результирующая амплитуда намного более гладкая функция.


	\section{Базис волновых пакетов.}
Рассмотрим процесс рассеяния двух частиц. Гамильтониан этой системы запишем выделив взаимодействие и гамильтониан свободной частицы:
\[
	H = H_0 + V,
\]
где потенциал $V$ – короткодействующий с характерным размером $a$. Собственные функции $H_0$ это состояния с определенным волновым вектором, выберем их нормированными на дельта-функцию:
\[
	\ket{\vec{k}} = \frac{1}{(2\pi)^{3/2}}e^{i\vec{k}\vec{r}} 
\]\[
	\braket{\vec{k}|\vec{k'}} = \delta(\vec{k} - \vec{k'})
\]

	\subsection{Волновые пакеты.}
Перейдем от точных волновых функций к волновым пакетам. Введем разбиение $D$ трехмерного пространства волновых векторов на малые области и перенумеруем эти области индексом $\alpha$. Волновым пакетом мы будем называть сумму всех $\ket{\vec{k}}$, волновой вектор которых попадает в заданную область $D_\alpha$. При этом интегрирование произведем с весовой функцией $f(k)$ смысл которой будет объяснен далее:
\[
	\ket{\alpha} = \frac{1}{\sqrt{\mu_\alpha}} \int\limits_{D_\alpha} f(k) \ket{\vec{k}} d^3k
\]
Коэффициент $\mu_\alpha$ найдем из условия нормировки:
\[
	\braket{\alpha|\beta} = \frac{1}{\sqrt{\mu_\alpha \mu_\beta}}  \int\limits_{D_\alpha}  \int\limits_{D_\beta} f(k)f^\dagger(k') \delta(\vec{k} - \vec{k'}) d^3k d^3k' = 1
\]
Отсюда имеем:
\[
	\mu_\alpha =  \int\limits_{D_\alpha} |f(k)|^2 d^3k
\]
Таким образом мы перешли от непрерывного спектра к дискретному и полученный набор функций является ортонормированным и полным:
\[
	\braket{\alpha|\beta} = \delta_{\alpha,\beta}
\]\[
	\sum\limits_\alpha \ket{\alpha}\bra{\alpha} = 1
\]

Сетку разбиения удобно выбирать так, чтобы её поверхности уровня совпадали с поверхностями уровня системы координат в которых производится интегрирование. Например, если $d^3k = dk_xdk_ydk_z$, то логично выбрать прямоугольную сетку. Более же удобным в данном случае будет интегрирование в сферической системе координат, во-первых энергия зависит только от одной координаты – модуля волнового вектора, во-вторых явно выделен азимутальный угол и это существенно упростит решение задач со сферически-симметричным потенциалом. Таким образом $D_\alpha$ это следующая область:
\[
 D_{\alpha(i,j,k)} = \big\{ \: (k,\theta,\varphi) \: \big| \: k \in [k_i,k_{i+1}], \theta \in [\theta_j,\theta_{j+1}], \varphi \in [\varphi_k,\varphi_{k+1}] \: \big\} 
\]\[
 \alpha \text{ номерует сочетания } \big\{ (0,0,0), \; ... \; , (i_{max},j_{max},k_{max}) \big\}
\]

	\subsection{Волновые пакеты в аксиально-симметричных системах.}
В случае сферически-симмеричного потенциала результат решения задачи не должен зависеть от азимутального угла. При этом удобно смешанное представление, в котором угол $\varphi$ – непрерывная переменная, а модуль $\vec{k}$ и полярный угол – дискретные. Интегрирование проведем в сферических координатах. Итого, имеем следующее представление:

\[
	\ket{\alpha,\varphi} = \frac{1}{\sqrt{\mu_\alpha}} \int\limits_{D_\alpha} f(k) \ket{\vec{k}} k^2 \,dk \,d(-\cos\theta)
\]
\[
	\mu_\alpha =  \int\limits_{D_\alpha} |f(k)|^2 k^2 \,dk \,d(-\cos\theta)
\]
Определенные так волновые пакеты нормированны следующим условием:
\[
	\braket{\alpha,\varphi|\beta,\varphi'} = \delta_{\alpha,\beta}\delta(\varphi'-\varphi)
\]\[
	\sum\limits_\alpha \int\limits_0^{2\pi} \,d \varphi \ket{\alpha,\varphi}\bra{\alpha,\varphi} = 1
\]
Теперь рассмотрим подробнее сетку:
\[
 \text{ узлы сетки по }k: \;\; k_i, \;\;\; i=\overline{0,M}
\]\[
 \text{ узлы сетки по }\theta: \;\;  \theta_j, \;\;\; j=\overline{0,N}
\]
На каждом интервале выберем средний элемент
\[
 k_i^* \in [k_i,k_{i+1}], \;\;\; i=\overline{0,M-1}
\]\[
 \theta_j^* \in [\theta_j,\theta_{j-1}], \;\;\; j=\overline{0,N-1}
\]
и перейдем от двойного индекса $(i,j)$ к одинарному
\[
	\alpha = \overline{0,N M-1},
\]\[
	i = \alpha \bmod N, \;\;\; j = [\alpha/N],
\]\[
	\vec{k^*_\alpha} = (k_i^*,\theta_j^*),
\]
где $x \bmod y$ – остаток от деления $x$ на $y$, а $[x/y]$ – целая часть деления. Если рассматривать величины $(k_i^*,\theta_j^*)$ как элементы матрицы, то такое преобразование эквивалетно следующему изменению индексов в матрице:
\[
\begin{pmatrix} 
a_{00}    & a_{01}    & \dots  & a_{0,N-1}  \\ 
a_{10}    & a_{11}    & \dots  & a_{1,N-1}  \\ 
\vdots    & \vdots    & \ddots & \vdots     \\ 
a_{M-1,0} & a_{M-1,1} & \dots  & a_{M-1,N-1}  
\end{pmatrix}  \longrightarrow  \begin{pmatrix} 
a_{0}      & a_{1}        & \dots   & a_{N-1}  \\  
a_{N}      & a_{N+1}      & \dots   & a_{2N-1} \\ 
\vdots     & \vdots       & \ddots  & \vdots   \\
a_{(M-1)N} & a_{(M-1)N+1} & \dots   & a_{NM-1} 
\end{pmatrix}
\]

\subsection{Операторы в представлении ВП}

Теперь получим выражение операторов в представлении ВП через операторы в импульсном представлении. Пусть дан оператор
\[
	A(\vec{k'},\vec{k}) \equiv \bra{\vec{k'}}A\ket{\vec{k}}
\] и оператор \[
  A_{\alpha\beta}(\varphi',\varphi) \equiv \bra{\alpha,\varphi'}A\ket{\beta,\varphi}.
\]
Переход от одного представления к другому осуществляется по стандартной формуле:
\[
	A_{\alpha\beta}(\varphi',\varphi) = \int d^3k \int d^3k' \braket{\alpha,\varphi'|\vec{k'}} A(\vec{k'},\vec{k}) \braket{\vec{k}|\beta,\varphi}.
\] 
Перекрытие ВП и плоской волны вычисляется исходя из определения ВП:
\[
 \braket{\vec{k}|\alpha,\varphi'} = \frac{f(k)}{\sqrt{\mu_\alpha}} \delta_{\alpha,\{\vec{k}\}} \delta(\varphi'-\varphi)
\]
Здесь и далее под $\{\vec{k}\}$ подразумевается номер интервала в который попадает этот вектор. Подставляя это в интеграл перехода получим:
\[
 A_{\alpha\beta}(\varphi',\varphi) = \frac{1}{\sqrt(\mu_\alpha \mu_\beta)} \int\limits_{D_\alpha} f^\dagger(k) d^2k \int\limits_{D_\beta} f(k') d^2k' A(\vec{k'},\vec{k})
\] Пользуясь теоремой о среднем и малостью интервала интегрирования выносим $A(\vec{k'},\vec{k})$ за знак интеграла:
\[
 A_{\alpha\beta}(\varphi',\varphi) = \frac{A(\vec{k^*_\alpha},\vec{k^*_\beta})}{\sqrt{\mu_\alpha \mu_\beta}} \int\limits_{D_\alpha} f^\dagger(k) d^2k \int\limits_{D_\beta} f(k') d^2k'
\] Оставшиеся два интеграла также пользуясь теоремой о среднем можно привести к виду нормировочных интегралов. И окончательный результат:
\[
 A_{\alpha\beta}(\varphi',\varphi) = \frac{\sqrt{\mu_\alpha \mu_\beta}}{ f^\dagger(k_\alpha^*)f(k_\beta^*) }  A(\vec{k^*_\alpha},\vec{k^*_\beta}).
\]
Все операторы рассеяния будут выражаться через велечины $A(\vec{k^*_\alpha},\vec{k^*_\beta})$ поэтому введем для них краткие обозначения:
\begin{equation}
    \label{mesh_oper}
    A(\vec{k^*_\alpha},\vec{k^*_\beta}) \equiv A^*_{\alpha\beta}
\end{equation}

\subsection{Поведение ВП в координатном пространстве.}
-- Рассмотреть при разных $f(k)$ --




\section{Уравнение Липпмана-Швингера.}

Двухчастичное рассеяние описывается уравнением Липпмана-Швингера
\begin{equation}
    T = V + V G_0 T,
\end{equation}
где $V$ — двухчастичный потенциал, $G_0 = (z - H_0)^{-1}$ — функция Грина свободной частицы и $T$ это T-матрица. В импульсном пространстве матричные элементы $T(\vec{q'},\vec{q},z) \equiv \bra{\vec{q'}}T(z)\ket{\vec{q}}$  удовлетворяют интегральному уравнению
\begin{equation}
    T(\vec{q'},\vec{q}, z) = V(\vec{q'},\vec{q}) + \int d^3q'' V(\vec{q'},\vec{q''}) G_0(\vec{q''},z) T(\vec{q''},\vec{q}, z). 
\end{equation}
Здечь $\vec{q}$ это относительный импульс, $m$ — приведенная масса двух частиц и $z$ это энергия. Мы рассматриваем нерелятивистский случай и ограничемся двумя бесспиновыми частицами. Таким образом $V(\vec{q'},\vec{q})$ и $T(\vec{q'},\vec{q},z)$ являются скалярными функциями:
\begin{equation}
    V(\vec{q'},\vec{q}) = V(q',q,\vec{q'}\vec{q})
\end{equation}
и
\begin{equation}
    T(\vec{q'},\vec{q}) = T(q',q,\vec{q'}\vec{q})
\end{equation}
В последнем выражении мы отбосили параметрическую зависимость от $z$. Рассмотрим теперь как выражаются операторы в представлении волновых пакетов через операторы в импульсном представлении.


\subsection{Функция Грина в представлении ВП.}
Действуя на функцию Грина свободной частицы
\[
	G(k_0) = \frac{2m}{\hbar^2} \int d^3k \frac{ \ket{\vec{k}} \bra{ \vec{k} } }{ k_0^2 - k^2 + i\epsilon } 
\]
операторами проектирования
\[
	P = \sum\limits_\alpha \int\limits_0^{2\pi} \,d \varphi \ket{\alpha,\varphi}\bra{\alpha,\varphi}
\] 
с двух сторон, получим функцию Грина в представлении ВП:

\[
  G(k_0) = 
	\frac{2m}{\hbar^2}  \sum\limits_\alpha \frac{1}{ \mu_\alpha } \int\limits_{D_\alpha} d^3k 
		\frac{ \ket{\alpha,\varphi} |f(k)|^2 \bra{\alpha,\varphi} }{ k_0^2 - k^2 + i\epsilon } 
\]
Рассматривая $\alpha$-ый элемент этой суммы, вводя сферические координаты и интегрируя по $\varphi$ и $\theta$ имеем:
\[
  G_\alpha = \frac{2m (cos(\theta_j^*)-cos(\theta_{j+1}^*))}{\hbar^2 \mu_\alpha} \int\limits_{\Delta k_i}
		\frac{ |f(k)|^2 k^2 dk }{ k_0^2 - k^2 + i\epsilon } 
\]
и окончательно в случае $f(k)=1$:
\begin{equation}
  \label{greene1}
  G_\alpha = \frac{2m}{\hbar^2} \frac{cos(\theta_j^*)-cos(\theta_{j+1}^*)}{\mu_\alpha}
		\bigg( 
			k_{i+1}^* - k_i^* + \frac{k_0}{2}\ln\frac{(k_{i+1}^* +k_0)(k_i^* -k_0)}{(k_{i+1}^* -k_0)(k_i^* +k_0)} + \frac{i\pi k_0}{2}\delta_{\{k'\},i}
		\bigg)
\end{equation}
и в случае $f(k)=1/k$:
\begin{equation}
  \label{greene2}
  G_\alpha = \frac{2m}{\hbar^2} \frac{cos(\theta_j^*)-cos(\theta_{j+1}^*)}{\mu_\alpha} 
		\bigg( 
			\frac{k_0}{2}\ln\frac{(k_{i+1}^* +k_0)(k_i^* -k_0)}{(k_{i+1}^* -k_0)(k_i^* +k_0)} + \frac{i\pi k_0}{2}\delta_{\{k'\},i}
		\bigg)
\end{equation}
Так же как и для операторов в формуле (\ref{mesh_oper}) для функции Грина введем обозначения
\begin{equation}
    \label{mesh_greene}
    G_\alpha = \frac{cos(\theta_j^*)-cos(\theta_{j+1}^*)}{\mu_\alpha} G^*_\alpha
\end{equation} где $G^*_\alpha$ можно найти сравнивая с формулами (\ref{greene1}) и (\ref{greene2}).


\subsection{Уравнение Липпманна-Швингера в представлении ВП.}

Используя результаты пункта 2.3 запишем в представлении ВП Т-матрицу:
\begin{equation}
	T_{\alpha\beta}(\varphi,\varphi') = \frac{\sqrt{\mu_\alpha \mu_\beta}}{ f^\dagger(k_\alpha^*)f(k_\beta^*) } T(\vec{k^*_\alpha},\vec{k^*_\beta})
\end{equation}
и аналагично потенциал.

Запишем теперь уравнение Липпманна-Швингера в представлении ВП:
\begin{equation}
 T_{\alpha\alpha_0}(\varphi-\varphi_0) = 
 V_{\alpha\alpha_0}(\varphi-\varphi_0) + 
 \sum\limits_\beta \int\limits_0^{2\pi} d \varphi' V_{\alpha\beta}(\varphi-\varphi') G_{\beta} T_{\beta\alpha_0}(\varphi'-\varphi_0)
\end{equation}
где $\alpha_0$ и $\varphi_0$ - координаты $\vec{k_0}$.  Так как потенциал сферически-симметричный Т-матрица не может зависить от $\varphi$. Зависимость от этого угла есть только в потенциале и имеет следующий вид:
\begin{equation}
\label{pifagor_3d}
    V(\vec{q'},\vec{q}) = V(q',q, \cos(\theta')\cos(\theta) + \sin(\theta')\sin(\theta)\cos(\varphi) )
\end{equation} 
Поэтому можем провести интегрирование по $\varphi$ и ввести новую функцию:
\begin{equation}
    \label{phi_int}
    W_{\alpha\beta} = \int\limits_{0}^{2\pi} d\varphi V_{\alpha\beta}(\varphi)
\end{equation}
Если один из индексов отвечает импульсу налетающей частицы, то можно проинтегрировать явно, так как зависимость от $\varphi$ в выражении (\ref{pifagor_3d}) исчезает.
\begin{equation}
    V_{\alpha\alpha_0} = \frac{1}{2\pi}W_{\alpha\alpha_0}
\end{equation}

Тогда уравнение принимает вид
\begin{equation}
    T_{\alpha\alpha_0} = \frac{1}{2\pi}W_{\alpha\alpha_0} + \sum\limits_\beta  W_{\alpha\beta} G_{\beta} T_{\beta\alpha_0}
\end{equation}
Используя формулы (\ref{mesh_oper}), (\ref{mesh_greene}) и сокращая нормировочные множители, получим явный вид данного уавнения:
\begin{equation}
    T^*_{\alpha\alpha_0} = \frac{1}{2\pi}W^*_{\alpha\alpha_0} +
 \sum\limits_\beta \frac{cos(\theta_j^*)-cos(\theta_{j+1}^*)}{|f_\beta|^2} W^*_{\alpha\beta} G^*_{\beta} T^*_{\beta\alpha_0}
\end{equation}

Решив уравнение Липпманна-Швингера получим матричные элементы Т-оператора. Они прямо пропорциональны амплитуде рассеяния. Учитывая сохранение импульса имеем:
\begin{equation}
    f_t = - \frac{2m}{\hbar^2} \frac{(2\pi)^2}{2} T^*_{\alpha\alpha_0},
\end{equation}
где $\alpha \in {\alpha_0, ... , \alpha_0 + N}$, а $t \in {0, ... ,N} $.




%\section{Поведение высших борновских членов}
\section{Численные иллюстрации}
Для иллюстрации полученного метода произведен расчет сечения рассеяния для гауссового потенциала и потенциала Малфлье-Тьёна. В обоих случаях результаты сравнивались с методом парциальных волн. 


	\subsection{Расчет для Гауссова потенциала.}
	В качестве простого теста используем потенциал в виде гауссового пика.
	\begin{equation}
	   V(r) = A e^{-r^2/a^2},
	\end{equation}
	и соостветственно
	\begin{equation}
	   V(\vec{q'},\vec{q}) = \frac{Aa^3}{8\pi^{3/2}}\exp\big( - \frac{a^2}{4}(\vec{q'}-\vec{q})^2 \big)
	\end{equation}
	Интегрирование по $\varphi$ в соответствии с (\ref{phi_int}) может быть проведено аналитически
	\begin{equation}
	   W(\vec{q_2},\vec{q_1}) = \frac{Aa^3}{4\pi^{1/2}} e^{ - a^2( q_1^2 + q_2^2 - 2q_1q_2\cos\theta_1\cos\theta_2)/4 }
	    I_0 \big( a^2q_1q_2\sin\theta_1\sin\theta_2/2 \big)
	\end{equation}
	где $I_0$ -- модифицированная функция Бесселя.
	
	Результаты сравним с методом волновых функций.
	
	
	
	\subsection{Расчет для потенциала Юкавы.}


\end{document}

